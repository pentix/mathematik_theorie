\documentclass[12pt,a4paper,oneside]{scrbook}
\usepackage[utf8]{inputenc}
\usepackage[ngerman]{babel}
\usepackage{amsmath}
\usepackage{amsfonts}
\usepackage{amssymb}
\usepackage{graphicx}
\usepackage[hidelinks]{hyperref}
\author{\{Autoren hier vllt. einfügen\}}
\title{Mathematik 1. Kantonsschule}
\begin{document}


\maketitle
\tableofcontents

\newpage

\chapter{Potenzgesetze}


\chapter{Strahlensätze und Ähnlichkeit}


\chapter{Funktionen}
\section{Definition einer Funktion}
Die Definition erfolgt durch eine Deklaration inklusive
der Zahlbereichen und des Zahlbereiches des "Rückgabewertes" der Funktion:

\[ f : \mathbb{R} \rightarrow \mathbb{R}\]
\[ f(x) = \frac{3}{7} + a^2 - \frac{b}{3x} + 2; \quad\quad x \in \mathbb{Z}\backslash\{0\}\]

\section{Lineare Funktion}
\subsection{Typische Definition}
\[f(x) = y = k \cdot x + d; \quad\quad k, d \in \mathbb{R}\]
\subsection{Lösungsformel}
\[x = \frac{-d}{k}\]

\subsection{Beispiel}
\[f(x) = -\frac{3}{2}x + 6\]

\subsection{Wertetabelle} der Funktion $f(x)$ wobei $x \in [-3; 7]$\\\\
\begin{tabular}{l||c|c|c|c|c|c|c|c|c|c|c}
$x$ & -3 & -2 & -1 & 0 & 1 & 2 & 3 & 4 & 5 & 6 & 7\\
\hline
$f(x)$ & 10.5 & 9 & 7.5 & 6 & 4.5 & 3 & 1.5 & 0 & -1.5 & -3 & -4.5\\
\end{tabular}\\\\\\
\textbf{Beachte:} $\Delta x$ ist konstant $-1.5$.\\

\subsection{Steigung}
\[f(x) = k\cdot x + d; \quad\quad k,d \in \mathbb{R}\]
$d$ ist der y-Achsenabschnitt. An diesem Ort wird die y-Achse\\
geschnitten. $k$ ist die Steigung von f.\\\\\\
\begin{tabular}{ll}
Wenn $k \geq 0$: & f ist wachsend\\
Wenn $k < 0$: & f ist fallend\\
\end{tabular}

\subsection{Stufenformel}
Herleitung der Stufenformel:
\[f(x+1) = f(x) + k\]
\[f(x+1) - f(x) = k\]
\[f(x+\Delta x) = f(x) + \Delta x \cdot k\]
\[f(x+\Delta x) - f(x) = \Delta x \cdot k\]
\[f(x+\Delta x) - f(x) = k\]

\begin{center}
\fbox{\parbox{4cm}{\[\frac{\Delta y}{\Delta x} = k\]}}
\end{center}


\subsubsection{Deutung der Steigung k}
Die Steigung einer linearen Funktion entspricht der Änderung\\
der Funktionswerte, bei Vermehrung des Arguments um 1.

\section{Quadratische Funktion}
\subsection{Typische Definition}
\[f(x) = y = ax^2 + bx + c; \quad\quad a \in \mathbb{R}\backslash\{0\}; \;\; b, c \in \mathbb{R}\]
\subsection{Lösungsformel}
\[x_1 / x_2 = \frac{-b \pm \sqrt{b^2-4ac}}{2a} \]
\subsection{Beispiel}
\[f(x) = \frac{1}{2}x^2 - \frac{3}{4}x - \frac{7}{2}\]

\subsection{Wertetabelle} der Funktion $f(x)$ wobei $x \in [-3; 7]$\\\\
\begin{tabular}{l||c|c|c|c|c|c|c|c|c|c|c}
$x$ & -3 & -2 & -1 & 0 & 1 & 2 & 3 & 4 & 5 & 6 & 7\\
\hline
$f(x)$ & 3.25 & 0 & -2.25 & -3.5 & -3.75 & -3 & -1.25 & 1.5 & 5.25 & 10 & 15.75\\
\end{tabular}\\\\\\
\textbf{Beachte:} $\Delta x$ ist nicht konstant!\\


\chapter{Quadratische Gleichungen}


\chapter{Lineare Gleichungssysteme}


\chapter{Logik}
\section{Aussagen}
\section{Logikoperatoren}

\chapter{Beweise}
\section{Direkte Beweise}

\section{Indirekte Beweise}

\section{Induktive Beweise}

\chapter{Darstellende Geometrie}
\section{Kavaliersprojektion}
\section{Zweitafeldarstellung}
\section{Dreitafeldarstellung}

\chapter{Komplexe Zahlen}
\section{Grundlagen}

\section{Kartesische Darstellung}

\section{Polardarstellung}

\section{Rechenoperationen}
\subsection{Addition}
\subsection{Subtraktion}
\subsection{Multiplikation}
\subsection{Division}
\subsection{Potenzen}

\section{Quadratische Gleichungen}

\section{Lineare Gleichungssysteme}


\end{document}
